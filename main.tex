\documentclass[11pt]{scrartcl}
\usepackage{graphicx}
\usepackage[utf8]{inputenc}
\usepackage{amsmath}
\usepackage{listings}

\title{1. Report Multicore Systems}
\author{Group 9: Simon Lermen, Ming Hong Li}

\date{\today}

\begin{document}
\maketitle
\section{Set Associative Cache}

\subsection{Calculations}

The tag, index and offset where computed in the following way:
\begin{lstlisting}
int addr   = Port_Addr.read();
int offset = (addr & 31);
int tag = (addr >> 5);
int index = (tag & 127);     
tag = (tag >> 7); 
\end{lstlisting}

\subsection{Results}

Figure~1 shows a snapshot from the waveform file generated by the program.
The hitrate varied significantly between the provided tracefiles. The tracefile dbg\_p1.trf produced adresses that were much smaller than 32~bit resulting in the tag being equal to 0 most of the time. This can explain the high hitrate compared to rnd\_p1.trf in table~\ref{table:bench}. 

\begin{table}
\begin{tabular}{|c|c|c|c|c|c|c|c|c|}
\hline
Tracefile & Duration &	Reads &	RHit &	RMiss &	Writes &	WHit &	WMiss	& Hitrate \\
\hline
dbg\_p1.trf & 5~ms	& 40	& 39&	1&	60	&57&	3&	96.000000 \\
rnd\_p1.trf & 2788~ms &	33031 &	18714 &	14317	&32505&	18368&	14137	&56.582642 \\
\hline

\end{tabular}
\caption{Results for Benchmarks}
\label{table:bench}
\end{table}


\begin{figure}
\centering
\includegraphics[scale=0.7]{./graphics/wavefilecut.pdf}

\caption{Snapshot from a generated waveform file}
\label{fig:waveform}
\end{figure}

\end{document}